\subsection*{B1}
    Le besoin d'une solution technique telle que les containers provient de plusieurs demandes:
    \begin{itemize}
        \item \textbf{La simplification de la configuration}: la plupart des applications sont déployées dans le cloud aujourd'hui, et les entreprises peuvent être amenées à changer de fournisseur d'infrastructure (IaaS) ou de plateforme (PaaS) durant leur vie. Ceci est en partie possible grâce aux machines virtuelles, mais leur consommation de ressources est très importante et la répétabilité de la construction de telles machines n'est pas souvent aisée;
        \item \textbf{Un environnement constant dans la chaîne de livraison}: de la machine d'un développeur à la production, le code s'exécute dans plusieurs environnements, parfois chez différents prestataires. Chaque environnement (développement, test, QA, staging, production\dots) possède sa part de changements mineurs qui peuvent avoir un impact sur le comportement de l'application finale. Ici le besoin est une infrastructure immutable, que l'on peut reproduire facilement;
        \item \textbf{Productivité des développeurs}: les développeurs ont besoin de deux choses principales : avoir un environnement de travail le plus proche possible de la production et pouvoir travailler rapidement dans un tel environnement, sans passer des heures à configurer des composants. Ceci permet à ce que des développeurs moins à l'aise avec l'administration système puissent effectuer leur travail et permet d'éliminer la difficulté de la mise en place d'un environnement de travail pour une nouvelle recrue;
        \item \textbf{Isolation des applications}: il arrive parfois que des applications doivent fonctionner en même temps et qu'elles utilisent les mêmes technologies, mais dans des versions différentes. Le besoin d'isolation se fait alors ressentir, pour éviter ce que l'on appelle souvent le \enquote{\textit{dependency hell}};
        \item \textbf{Gestion des ressources}: les machines virtuelles ont souvent été très efficaces pour déployer plusieurs composants sur un serveur physique hôte puissant. Toutefois, leurs consommations en ressource restent importantes et les ressources non utilisées ne peuvent pas être allouées à d'autres machines virtuelles, qui auraient besoin de plus de ressources au même moment;
        \item \textbf{Rapidité de déploiement}: auparavant, mettre en place un nouveau serveur physique et le configurer prenait plusieurs heures. La virtualisation a réduit ce temps à l'ordre de quelques minutes mais le besoin de réduire ce temps est encore présent. En effet l'utilisation de services est variable en fonction de la demande et parfois les ressources non utilisées pourraient ne pas être payées dans le cas d'IaaS ou de PaaS. Ces ressources non utilisées par la production pourraient également être affectées à la réalisation de tâches non urgentes, peu gourmandes ou longues en temps d'exécution. On retrouve ce besoin de changement d'échelle et d'affectation de ressources, avec une demande de réactivité et de rapidité d'exécution.
    \end{itemize}
