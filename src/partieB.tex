\subsection*{B1}
    \subsubsection*{Les besoins qui ont mené à l'émergence des containers}
        Le besoin d'une solution technique telle que les containers provient de plusieurs demandes:
        \begin{itemize}
            \item \textbf{La simplification de la configuration}: la plupart des applications sont déployées dans le cloud aujourd'hui, et les entreprises peuvent être amenées à changer de fournisseur d'infrastructure (IaaS) ou de plateforme (PaaS) durant leur vie. Ceci est en partie possible grâce aux machines virtuelles, mais leur consommation de ressources est très importante et la répétabilité de la construction de telles machines n'est pas souvent aisée;
            \item \textbf{Un environnement constant dans la chaîne de livraison}: de la machine d'un développeur à la production, le code s'exécute dans plusieurs environnements, parfois chez différents prestataires. Chaque environnement (développement, test, QA, staging, production\dots) possède sa part de changements mineurs qui peuvent avoir un impact sur le comportement de l'application finale. Ici le besoin est une infrastructure immutable, que l'on peut reproduire facilement;
            \item \textbf{Productivité des développeurs}: les développeurs ont besoin de deux choses principales : avoir un environnement de travail le plus proche possible de la production et pouvoir travailler rapidement dans un tel environnement, sans passer des heures à configurer des composants. Ceci permet à ce que des développeurs moins à l'aise avec l'administration système puissent effectuer leur travail et permet d'éliminer la difficulté de la mise en place d'un environnement de travail pour une nouvelle recrue;
            \item \textbf{Isolation des applications}: il arrive parfois que des applications doivent fonctionner en même temps et qu'elles utilisent les mêmes technologies, mais dans des versions différentes. Le besoin d'isolation se fait alors ressentir, pour éviter ce que l'on appelle souvent le \enquote{\textit{dependency hell}};
            \item \textbf{Gestion des ressources}: les machines virtuelles ont souvent été très efficaces pour déployer plusieurs composants sur un serveur physique hôte puissant. Toutefois, leurs consommations en ressources restent importantes et les ressources non utilisées ne peuvent pas être allouées à d'autres machines virtuelles, qui auraient besoin de plus de ressources au même moment;
            \item \textbf{Rapidité de déploiement}: auparavant, mettre en place un nouveau serveur physique et le configurer prenait plusieurs heures. La virtualisation a réduit ce temps à l'ordre de quelques minutes mais le besoin de réduire ce temps est encore présent. En effet l'utilisation de services est variable en fonction de la demande et parfois les ressources non utilisées pourraient ne pas être payées dans le cas d'IaaS ou de PaaS. Ces ressources non utilisées par la production pourraient également être affectées à la réalisation de tâches non urgentes, peu gourmandes ou longues en temps d'exécution. On retrouve ce besoin de changement d'échelle et d'affectation de ressources, avec une demande de réactivité et de rapidité d'exécution.
            \item \textbf{Versionnage et partage de l'infrastructure}. Les gestionnaires de version sont désormais habituels dans le monde de l'ingénierie logicielle. Se pose alors la question: pourquoi n'est-il pas possible de pouvoir partager des éléments d'infrastructure, de description des composants, des services des paquets dans des fichiers de configuration. On parle souvent d'\enquote{\textit{Infrastructure as code}}. Comme pour les gestionnaires de version, l'envie de pouvoir télécharger des descriptions d'infrastructure créées par d'autres personnes se faire sentir, pour éviter que chacun ne réinvente la roue de son côté.
        \end{itemize}

     \subsubsection*{Approche technique des containers}
        Pour répondre au besoin de déploiement rapide et à la consommation faible en ressources, les containers prennent le contre pied des pieds virtuelles et ne cherchent pas à reproduire une machine physique.\\

        Les machines virtuelles permettent de simuler une machine physique et donc de faire tourner une application de très bas niveau à savoir un système d'exploitation de son choix. Grâce à cela, une machine physique sous un système d'exploitation peut faire tourner plusieurs machines virtuelles avec chacun son propre système d'exploitation et son ensemble applicatif. Cette solution est relativement performante si elle repose sur des fonctions matérielles (architecture physique compatible et si possible avec instruction AMD-V, VT-d, VT-x\dots) car sinon il faut opérer par émulation (perte de près de 50\% des performances CPU lors d'émulation de x86 sur PowerPC)\cite{ipponDocker}. Les machines virtuelles sont souvent considérées comme sécurisées car il n'y a pas de communication directe entre machine virtuelles et la machine hôte.\\

        Un container virtualise l'environnement d'exécution de le système d'exploitation de la machine hôte (Linux ou BSD, il n'existe pas à ce jour de container sous Windows). Un container est un ensemble applicatif s'exécutant au sein de du système d'exploitation maître de manière virtuellement isolé et contraint. Le container est très performant et léger car il partage de nombreuses ressources avec le système d'exploitation hôte (kernel, devices\dots). En revanche bien que s'exécutant de manière isolée, le container ne peut être considéré comme très sécurisé puisque partageant la stack d'exécution avec le système d'exploitation maître. Le container peut au choix démarrer un système d'exploitation complet ou bien simplement des applications. D'autres outils utilisant des containers tel qu'OpenStack ou Proxmox gèrent des containers pour virtualiser tout le système d'exploitation.\\

        Pour que la configuration et le partage de ces containers soient pratiques, la description du contenu d'un container est souvent réalisée dans un fichier texte. Un programme est alors requis pour créer un container à partir du fichier. L'utilisation d'un fichier texte de description d'un container facilite le versionnage de celui-ci dans un gestionnaire de version. Les logiciels de containers proposent souvent des annuaires de container, où la communauté open source peut partager des containers déjà créés, que l'on peut configurer avec ses besoins par la suite.\\

        Enfin, même s'il est souvent possible de définir quelques éléments de configuration dans les containers (système d'exploitation, paquets à installer, utilisateurs à créer), ceux-ci ne cherchent pas à remplacer des gestionnaires de configuration comme Chef, Puppet ou Ansible qui offrent bien plus de possibilités.
